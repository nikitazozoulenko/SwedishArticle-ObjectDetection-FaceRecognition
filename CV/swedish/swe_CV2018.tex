%%%%%%%%%%%%%%%%%%%%%%%%%%%%%%%%%%%%%%%%%
% "ModernCV" CV and Cover Letter
% LaTeX Template
% Version 1.11 (19/6/14)
%
% This template has been downloaded from:
% http://www.LaTeXTemplates.com
%
% Original author:
% Xavier Danaux (xdanaux@gmail.com)
%
% License:
% CC BY-NC-SA 3.0 (http://creativecommons.org/licenses/by-nc-sa/3.0/)
%
% Important note:
% This template requires the moderncv.cls and .sty files to be in the same 
% directory as this .tex file. These files provide the resume style and themes 
% used for structuring the document.
%
%%%%%%%%%%%%%%%%%%%%%%%%%%%%%%%%%%%%%%%%%

%----------------------------------------------------------------------------------------
%   PACKAGES AND OTHER DOCUMENT CONFIGURATIONS
%----------------------------------------------------------------------------------------

\documentclass[10pt,a4paper,sans]{moderncv} % Font sizes: 10, 11, or 12; paper sizes: a4paper, letterpaper, a5paper, legalpaper, executivepaper or landscape; font families: sans or roman

\usepackage[T1]{fontenc}
\usepackage[utf8]{inputenc}
\usepackage[swedish]{babel}

\moderncvstyle{classic} % CV theme - options include: 'casual' (default), 'classic', 'oldstyle' and 'banking'
\moderncvcolor{blue} % CV color - options include: 'blue' (default), 'orange', 'green', 'red', 'purple', 'grey' and 'black'

\usepackage{lipsum} % Used for inserting dummy 'Lorem ipsum' text into the template

\usepackage[scale=0.78, top=2.5cm]{geometry} % Reduce document margins
%\usepackage[scale=0.75, top=0.5cm, bottom=0.5cm, left=0.5cm, right=0.5cm]{geometry} 
%\setlength{\hintscolumnwidth}{3cm} % Uncomment to change the width of the dates column
%\setlength{\makecvtitlenamewidth}{10cm} % For the 'classic' style, uncomment to adjust the width of the space allocated to your name

%----------------------------------------------------------------------------------------
%   NAME AND CONTACT INFORMATION SECTION
%----------------------------------------------------------------------------------------

\firstname{Nikita} % Your first name
\familyname{Zozoulenko} % Your last name

% All information in this block is optional, comment out any lines you don't need
\title{Curriculum Vitae}
\address{Tyrgatan 2}{Stockholm, Sverige 114 27}
\mobile{+46723053504}
\email{nikita.zozoulenko@gmail.com}
%\homepage{https://www.linkedin.com/pub/daniele-perrotta/64/5b9/794} {staff.org.edu/$\sim$jsmith} % The first argument is the url for the clickable link, the second argument is the url displayed in the template - this allows special characters to be displayed such as the tilde in this example
\photo[130pt][0.7pt]{portrait_orig.jpg} % The first bracket is the picture height, the second is the thickness of the frame around the picture (0pt for no frame)
%\quote{"A witty and playful quotation" - John Smith}

%----------------------------------------------------------------------------------------

\begin{document}

\makecvtitle % Print the CV title

%----------------------------------------------------------------------------------------
%   EDUCATION SECTION
%----------------------------------------------------------------------------------------

\section{Utbildning}
\cventry{2018--nutid}{Kungliga Tekniska Högskolan (KTH)}{Teknisk fysik, civilingenjör}{}{}{}
%2018--present

\cventry{2015--2018}{Katedralskolan i Linköping}{Naturvetenskapsprogrammet (NANAT)}{Under mitt tredje år läste jag dessutom matematik på Linköpings Universitet}{}{}

\cventry{2011--2015}{Internationella Engelska Skolan i Linköping}{Årskurs 6--9}{}{}{}


%\section{Masters Thesis}

%\cvitem{Title}{\emph{Technologies and characterization of ferroelectric polymers for biomedical sensors}}
%\cvitem{Supervisors}{Professor Antonino Fiorillo}
%\cvitem{Description}{This thesis is based on the implementation of a temperature sensor.}

%----------------------------------------------------------------------------------------
%   WORK EXPERIENCE SECTION
%----------------------------------------------------------------------------------------

\section{Arbetslivserfarenhet}

\cventry{Juni 2018 -- Augusti 2018}{Maskininlärningsingenjör}{\textsc{ContextVision AB}}{Linköping}{}{Jag var anställd som forskningsingenjör och jobbade med bildanalys inom digital patologi för att detektera cancer. Där skapade jag nya modeller som behandlade instanssegmentering av körtlar på mikrometernivå. Under anställningen skrev jag dessutom en vetenskaplig artikel med namnet \textit{Gland Instance Segmentation Through Overlapping Contour Regions and Random Transformation Sampling}.}

\cventry{Juni 2017 -- Juli 2017}{Djurskötsel, trädgårdsskötsel och museiguide}{\textsc{Friluftsmuseet Gamla Linköping}}{Linköping}{}{Mina arbetsuppgifter var uppdelade i att ta hand om museets djur, att agera som trädgårdsmästare, och att vara museiguide.}

\cventry{Augusti 2016}{Ungdomskonsult}{\textsc{AB Stångåstaden}}{Linköping}{}{Jag arbetade som gruppledare med andra ungdomskonsulter för att hitta lösningar och innovativa id\'eer för ett långvarande problem bland Stångåstadens fastigheter.}




%------------------------------------------------

%----------------------------------------------------------------------------------------
%   COMPUTER SKILLS SECTION
%----------------------------------------------------------------------------------------

\section{Tekniska kompetenser}

\cvitem{Språk}{\textsc{Python, C++, Java, GLSL,} (Jag läser en kurs i \textsc{MATLAB} efter jul)}
\cvitem{OS}{\textsc{Linux, Windows}}
\cvitem{Frameworks}{\textsc{Pytorch, TensorFlow, OpenGL, \LaTeX}}


%----------------------------------------------------------------------------------------
%   COMMUNICATION SKILLS SECTION
%----------------------------------------------------------------------------------------

%\section{Communication Skills}

%\cvitem{2010}{Oral Presentation at the California Business Conference}
%\cvitem{2009}{Poster at the Annual Business Conference in Oregon}

%----------------------------------------------------------------------------------------
%   LANGUAGES SECTION
%----------------------------------------------------------------------------------------

\section{Språk}

\cvitemwithcomment{Svenska}{Flytande}{}
\cvitemwithcomment{Engelska}{Flytande}{}
\cvitemwithcomment{Ryska}{Goda översättningsförmågor i tal}{}
\cvitemwithcomment{Spanska}{Gymnasienivå, förstår enkla samtal och texter}{}
%\cvitemwithcomment{Dutch}{Basic}{Basic words and phrases only}Svenska: Modersmål

%----------------------------------------------------------------------------------------
%   INTERESTS SECTION
%----------------------------------------------------------------------------------------

%----------------------------------------------------------------------------------------
%   AWARDS SECTION
%----------------------------------------------------------------------------------------
\section{Utmärkelser}

\cventry{Mars 2018}{Vinnare i Finalen Utställning Unga Forksare}{\textsc{Sverige}}{}{}{De 550 skarpaste projektarbetena bland Sveriges gymnasieelever tävlade för att vinna stipendier av ett totalt värde på 500 000 kr. I tävlingen ställde jag upp med bidraget \textit{Dense Face Detection with Convolutional Neural Networks} och vann toppriset att representera Sverige i världens största vetenskapstävling i USA, Intel ISEF}

\cventry{Maj 2018}{Finalist i Intel Internation Science and Engineering Fair}{\textsc{USA}}{}{}{Intel ISEF är världens största internationella projektutställning och tävling för studenter inom åldrarna 15-20. Totalt hade 1,8 miljoner ungdomar deltagit i uttagningstävlingarna världen över för att de bästa 1800 skulle delta i finalen. Mina tilldelade priser var:
\begin{itemize}
\item 1:a pris av \textit{Association for the Advancement of Artificial Intelligence}
\item 3:e pris i kategorin \textit{Robotics and Intelligent Machines}
\item 4:e pris av \textit{Association for Computing Machinery}
\end{itemize}
}



\cventry{Juni 2018}{Tre Bröders stipendium}{\textsc{Linköping}}{}{}{Stipendium för gymnasiestudent för utmärkelser inom studier i matematik}

\cventry{Juni 2018}{Framtidsstipendiet av Östsvenska Handelskammaren}{\textsc{Linköping}}{}{}{Stipendium för utvecklandet av en matematisk modell för ansiktsdetektering i stora folkmassor}

\section{Personliga Projekt}

\cventry{Oktober 2018}{AlphaZero Pytorch Implementation}{}{}{}{Jag implementerade, med fri källkod, en single-trådad version av Googles self play reinforcement learning agent med Monte Carlo Tree Search (MCTS) i PyTorch. Det är en modell som enbart genom spel mot sig själv blir bättre än en människa i spel med två spelare. Modellen användes för att skapa en AI som spelar $n \times n$ tic-tac-toe med godtyckliga storlekar och spelregler.}

\cventry{Augusti 2018}{Rubiks Kub-Agent}{}{}{}{Mitt mål var att utan någon mänsklig kunskap lära en dator lösa en rubiks kub. Det gjorde jag genom att träna en reinforcement learning agent att förutse resultatet av en Monte Carlo Tree Search för en godtyckligt given position med hjälp av Q-värden. Den slutgiltiga modellen kan lösa vilken $2 \times 2$ kub som helst på 2,5 sekunder.}

\cventry{Juli 2018}{Gland Instance Segmentation Through Overlapping Contour Regions and Random Transformation Sampling}{}{}{}{Det var en vetenskaplig artikel jag skrev under min anställning på ContextVision AB. Jag utvecklade en algoritm för instanssegmentering av körtlar på mikrometernivå i hematoxylin-eosin-färgningar av medicinska bilder inom digital patologi.}

\cventry{Maj 2018}{Improving Temporal Convolutional Networks}{}{}{}{Jag förbättrade tidsenliga faltningsnätverk, jämförde dem med de traditionella \textit{long short-term memory} LSTM-nätverken och programmerade en modell för automatisk bildtextgenerering.}

\cventry{Mars 2018}{Real-time Style Transfer}{}{}{}{Ett personligt grupprojekt skrivet i Python och JavaScript där målet var att omvandla ett live videoflöde till en godtycklig given konstnärlig stil. Vi använde oss av style transfer, neurala faltningsnätverk och en hemsida som GUI.}

\cventry{Mars 2018}{Dense Face Detection}{}{}{}{Mitt projekt i Utställningen Unga Forskare där jag använde mig av \textit{state of the art} objektdetekteringsalgoritmer för att skapa en matematisk modell för ansiktsdetektering i stora folkmassor.}

\section{Övrigt}
Jag tror att en balanserad livsstil är viktigt i livet och att ens personliga utveckling är minst lika viktig som sin akademiska utveckling. Det är därför jag utanför studierna sjunger i en kör och tränar minst tre gånger i veckan. Mina intressen inkluderar musik, träning, artificiell intelligens, naturresor och att träffa nya intressanta människor. 

\section{Referenser}
\begin{itemize}
\item \textbf{Martin Hedlund.}
Chief Technology Officer (CTO) och grundare av ContextVision AB \\
E-mail: martin.hedlund@contextvision.se \\
Telefon: +46 70 328 09 52

\vspace{10pt}

\item \textbf{Gabriel Skantze}, Professor i talteknologi och teknik vid Kungliga Tekniska Högskolan \\
E-mail: gabriel@speech.kth.se \\
Telefon: +46 8 790 78 74
\end{itemize}

%----------------------------------------------------------------------------------------
%   COVER LETTER
%----------------------------------------------------------------------------------------

% To remove the cover letter, comment out this entire block

%\clearpage

%\recipient{HR Department}{Corporation\\123 Pleasant Lane\\12345 City, State} % Letter recipient
%\date{\today} % Letter date
%\opening{Dear Sir or Madam,} % Opening greeting
%\closing{Sincerely yours,} % Closing phrase
%\enclosure[Attached]{curriculum vit\ae{}} % List of enclosed documents

%\makelettertitle % Print letter title

%\lipsum[1-3] % Dummy text

%\makeletterclosing % Print letter signature

%----------------------------------------------------------------------------------------

\end{document}