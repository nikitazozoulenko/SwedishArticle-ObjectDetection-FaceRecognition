\documentclass[11pt,a4paper]{article} % Font sizes: 10, 11, or 12; paper sizes: a4paper, letterpaper, a5paper, legalpaper, executivepaper or landscape; font families: sans or roman

\usepackage[T1]{fontenc}
\usepackage[utf8]{inputenc}
\usepackage[swedish]{babel}
\usepackage[a4paper, total={6in, 8in}]{geometry}

\setlength{\parindent}{0em}
\setlength{\parskip}{1em}
\begin{document}

%----------------------------------------------------------------------------------------
%   COVER LETTER
%----------------------------------------------------------------------------------------
\begin{center}
\textbf{Sammanfattning}
\end{center}

\begin{flushright}
\textit{Nikita Zozoulenko} \\
Tyrgatan 2\\
Stockholm 114 27, Sverige \\
\today
\end{flushright}
Hej,

Mitt namn är Nikita Zozoulenko och min resa började med en stort brinnande intresse för artificiell intelligens, matematik och Unga Forskare där jag vann toppriset att få representera Sverige internationellt i USA på Intel International Science and Engineering Fair. Jag bidrog med en maskininlärningsmodell för ansiktsigenkänning i stora folkmassor och juryns motivering lät som sådan: \textit{Fantastiskt arbete på doktorandnivå som dessutom är helt självständigt. Utställaren har härlett all matematik själv och implementerat den från grunden /.../}. Hela upplevelsen i tävlingen i USA hade en enorm påverkan på mig. För det första fick det mig att inse vilken potential vi har för framtiden genom att se 1000 innovativa projekt som fortsätter att driva mig än idag. För det andra var jag hedrad att ta emot tre priser, varav ett var förstaplats i världen inom utvecklingen av maskininlärning och artificiell intelligens. 

Efter resan har jag drivits av en lust för kunskap och utveckling, både karaktärsmässigt och akademiskt. Efter utställningen blev jag erbjuden att forska på KTH innan jag började studera där. På samma sommar blev jag anställd som forskningsingenjör på ContextVision där jag arbetade inom digital patologi för att detektera cancer på mikrometer-nivå i hematoxylin-eosin-färgningar av medicinska bilder. Under anställningen skrev jag dessutom en vetenskaplig artikel. Vid sidan av mina studier på universitetet har jag ägnat mig åt personliga projekt. Bland annat har jag skapat en maskininlärnings-agent som utan någon mänsklig kunskap lärt sig själv en algoritm för att lösa alla möjliga 2x2 Rubiks Kuber.  Jag har dessutom använt mig av samma matematisk teori som Google använde för att slå världsmästaren i brädspelet Go för att skapa en agent som kan slå människor i $N \times N$ tic-tac-toe, genom att endast spela mot sig själv. 

Jag söker detta stipendium för att ge stöd till min forskning. Stora projekt inom maskininlärning kräver hundratals datorer och det är något som hindrar min nuvarande forskning.

Tack,

Nikita Zozoulenko


%----------------------------------------------------------------------------------------

\end{document}